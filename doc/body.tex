\section{Opis problema}

\subsection{Zahtevi}

\subsubsection{Zadatak 1}
Potrebno je napisati dvoprolazni asembler za procesor opisan u nastavku. Ulaz
asemblera je tekstualni fajl sa izvornim kodom, u skladu sa sintaksom opisanom
u nastavku. Izlaz asemblera treba da bude predmetni program zapisan u
tekstualnom fajlu, po uzoru na predmetni program prikazan na vežbama.
Dozvoljeno je dodatno generisati i binarni fajl kao izlaz asemblera.
Prilikom generisanja izlaznog fajla voditi se principima koje koristi GNU
asembler, s tim da se sekcije smeštaju jedna za drugom, počev od adrese koja
se zadaje iz komandne linije prilikom pokretanja asemblera.

\subsubsection{Zadatak 2}
Potrebno je napisati interpretativni emulator za procesor opisan u nastavku.
Ulaz emulatora je izlaz asemblera, pri čemu je moguće zadati veći broj
predmetnih programa koje je potrebno učitati, povezati i pokrenuti. Nazivi
predmetnih programa se zadaju kao argumenti komandne linije. Dozvoljeno
je i odvojeno napisati linker, tako da linker radi povezivanje predmetnih
programa, a emulator kao argument prihvata jedan fajl koji treba da pokrene.
Emulacija je moguća samo ukoliko nakon učitavanja i povezivanja nema
preklapanja između učitanih fajlova, nema nerazrešenih simbola niti
višestru\-kih
definicija simbola i postoji definisan simbol START, koji definiše početnu
adresu prilikom izvršavanja.

\subsection{Opis okruženja}
Projekat se radi pod operativnim sistemom Linux, na x86 arhitekturi, koriste\-ći
programske jezike \textit{C}, \textit{C++} ili asemblerski jezik
(moguća je i kombinacija). Potrebni alati za programiranje su opisani u
materijalima sa predavanja i vežbi.

\subsection{Opis asemblerskog jezika}
Za sintaksu asemblera važi:
\begin{itemize}
    \item U jednoj liniji može biti najviše jedna komanda
          (instrukcija\\
          ili direktiva).
    \item Na početku svake linije može da stoji labela koju prati dvotačka
          (`\textbf{:}').
    \item Labela može da stoji i u praznoj liniji, što je ekvivalentno
          kao da stoji u liniji sa prvim sledećim sadržajem.
    \item Simboli se uvoze i izvoze tako što se na početku fajla navede
          direktiva \texttt{.global <ime\_globalnog\_simbola>, \ldots}
    \item Izvorni kod je podeljen u sekcije:
          \begin{itemize}
              \item \texttt{.text} -- sekcija koja sadrži mašinski kod
              \item \texttt{.data} -- sekcija koja sadrži inicijalizovane
                    podatke
              \item \texttt{.rodata} -- sekcija koja sadrži inicijalizovane
                    podatke koje nije dozvoljeno menjati
              \item \texttt{.bss} -- sekcija koja sadrži neinicijalizovane
                    podatke
          \end{itemize}
    \item Svaka sekcija se u fajlu može pojaviti najviše jednom.
    \item Fajl sa izvornim kodom se završava direktivom \texttt{.end}.
          Ostatak fajla se odbacuje.
      \item Direktive \texttt{.char}, \texttt{.word},
            \texttt{.long}, \texttt{.align} i \texttt{.skip} imaju iste
            funkcionalno\-sti kao u GNU asembleru.
\end{itemize}

\noindent Dodatno:
\begin{itemize}
    \item Komentar do kraja linije počinje znakom
          tačkazapeta (`\textbf{;}').
\end{itemize}

\noindent Sintaksa operanada instrukcija i argumenata direktiva:
\begin{itemize}
    \item Neposredna vrednost 20: 20
    \item Vrednost simbola x: \&x
    \item Memorijsko direktno adresiranje (labela x): x
    \item Vrednost sa lokacije sa adresom 20: *20
    \item Direktno registarsko adresiranje: r3
    \item Registarsko indirektno adresiranje sa pomerajem: r3\lbrack20\rbrack
    \item Registarsko indirektno adresiranje sa pomerajem: r3\lbrack x\rbrack
    \item PC relativno adresiranje sa pomerajem x: \$x
\end{itemize}

\subsection{Opis procesora}

\subsubsection{Osnovne informacije}
\begin{itemize}
    \item 16-bitni procesor sa Von-Neumann arhitekturom
    \item Ortogonalna, RISC arhitektura
    \item Adresibilna jedinica je bajt
    \item Raspored bajtova u reči je \textit{little-endian} (na nižoj adresi
          je niži bajt)
    \item Veličina adresnog prostora je $2^{16}B$
    \item Osam 16-bitnih opštenamenskih registara, označenih sa r0\ldots r7,
          pri čemu se r7 koristi kao PC (\textit{program counter}), a r6 kao SP
          (\textit{stack pointer})
    \item Pored opštenamenskih registara postoji i 16-bitni registar
          PSW\\ (\textit{program status word})
    \item Stek raste ka nižim adresama, a SP pokazuje na zauzetu lokaciju na
          vrhu steka
    \item Statusna reč PSW u najniža 4 bita sadrži flegove
          (redom od najnižeg bita):
          \begin{itemize}
              \item Z -- rezultat prethodne operacije je jednak nuli
              \item O -- rezultat prethodne operacije izazvao prekoračenje
              \item C -- rezultat prethodne operacije ima prenos
              \item N -- rezultat prethodne operacije je negativan
          \end{itemize}
    \item Najviši bit u PSW služi za globalno maskiranje prekida
\end{itemize}

\subsubsection{Prekidi}
Počev od adrese 0 nalazi se IVT (\textit{interrupt vector table}) sa 8 ulaza.
Svaki ulaz zauzima 2 bajta i sadrži adresu odgovarajuće prekidne rutine.
\begin{itemize}
    \item
    Na ulazu 0 se nalazi adresa prekidne rutine koja se izvršava prilikom
    pokretanja, odnosno prilikom resetovanja procesora.
    \item
    Na ulazu 1 se nalazi adresa prekidne
    rutine koja se izvršava periodično sa periodom od jedne sekunde. Prekidna
    rutina na ulazu 1 se izvršava periodično samo ako je bit 13 u PSW registru
    postavljen na vrednost 1, u suprotnom se ne izvršava.
    \item
    Na ulazu 2 se nalazi adresa prekidne rutine koja se izvršava prilikom
    pokušaja izvršenja nekorektne instrukcije.
    \item
    Na ulazu 3 se nalazi adresa prekidne rutine koja se izvršava kada se
    pritisne neki taster, i tada je potrebno očitati kod pritisnutog tastera.
\end{itemize}
Ostali ulazi su slobodni za korišćenje od strane
programera.

\subsubsection{Rad sa periferijama}
Poslednjih 128 bajtova adresnog prostora rezervisano je za periferije. Na
adresi 0xFFFE nalazi se registar podataka izlazne periferije. Upisom
\textit{ASCII} koda na adresu 0xFFFE na ekranu se na tekućoj poziciji
ispisuje odgovarajući znak, osim u slučaju upisa vrednosti 0x10, kada se
tekuća pozicija kursora premešta na početak sledećeg reda. Na adresi 0xFFFC
nalazi se registar podataka ulazne periferije. Kada se pritisne neki taster,
\textit{ASCII} kod pritisnut\-og tastera se upisuje u ovaj registar, nakon čega
sistem generiše prekid.

\subsubsection{Instrukcijski skup}
Sve instrukcije su veličine 2 bajta, izuzev instrukcija koje u sebi sadrže
neposredni podatak, apsolutnu ili PC relativnu adresu, kada je veličina
instr\-ukcije 4 bajta.

Sve instrukcije se izvršavaju uslovno, pri čemu je uslov
određen sa prva (najviša) 2 bita svake instrukcije. Naredna 4 bita rezervisana
su za operacioni kod instrukcije. Ostatak instrukcije podeljen je na dva polja
od po 5 bitova -- \textit{dst} i \textit{src}. Oba polja se kodiraju na isti
način, opisan u nastavku, s tim što \textbf{neposredno adresiranje nije
validan način adresiranja u polju \textit{dst}}.
Na mnemonike instrukcija dodaju se sufiksi koji označavaju uslov
koji treba da bude ispunjen kako bi se instrukcija izvršila. Najviša dva
bita instrukcije kodiraju koji od 4 uslova iz tabele je u pitanju.
\newpage % necessary evil

\begin{table}[h!]
\centering
\scriptsize

    \begin{tabu}{ | X[l] | X[l] | X[l] | X[l] | }
        \hline
        \cellcolor{blue!25}\textbf{\textit{Mnemonik}} &
        \cellcolor{blue!25}\textbf{\textit{Op. kod}} &
        \cellcolor{blue!25}\textbf{\textit{Objašnjenje}} &
        \cellcolor{blue!25}\textbf{\textit{Flegovi}} \\
        \hline
        \hline

        \textbf{add} & $0x0$ & $dst = dst + src$ & Z, O, C, N \\
        \hline

        \textbf{sub} & $0x1$ & $dst = dst - src$ & Z, O, C, N \\
        \hline

        \textbf{mul} & $0x2$ & $dst = dst \cdot src$ & Z, N \\
        \hline

        \textbf{div} & $0x3$ & $dst = dst \mathbin{/} src$ & Z, N \\
        \hline

        \textbf{cmp} & $0x4$ & $dst - src$ & Z, O, C, N \\
        \hline

        \textbf{and} & $0x5$ & $dst = dst\ \&\ src$ & Z, N \\
        \hline

        \textbf{or} & $0x6$ & $dst = dst\ |\ src$ & Z, N \\
        \hline

        \textbf{not} & $0x7$ & $dst = $\textasciitilde$ src$ & Z, N \\
        \hline

        \textbf{test} & $0x8$ & $dst\ \&\ src$ & Z, N \\
        \hline

        \textbf{push} & $0x9$ & mem\lbrack sp\ --=\ 2\rbrack=\textit{src}& \\
        \hline

        \textbf{pop} & $0xA$ & \textit{dst}=mem[sp], sp\ +=\ 2 & \\
        \hline

        \textbf{call} & $0xB$ & push pc, pc = \textit{src} & \\
        \hline

        \textbf{iret} & $0xC$ & pop psw, pop pc & \\
        \hline

        \textbf{mov} & $0xD$ & $dst = src$ & Z, N \\
        \hline

        \textbf{shl} & $0xE$ & $dst <<= src$ & Z, C, N \\
        \hline

        \textbf{shr} & $0xF$ & $dst >>= src$ & Z, C, N \\
        \hline
    \end{tabu}
    \caption{Pregled instrukcijskog skupa}
    \label{table:1}
\end{table}

\begin{table}[h!]
\centering
\small

    \begin{tabu}{ | X[l] | X[l] | X[l] | }
        \hline
        \cellcolor{blue!25}\textbf{\textit{Sufiks}} &
        \cellcolor{blue!25}\textbf{\textit{Kod}} &
        \cellcolor{blue!25}\textbf{\textit{Uslov}} \\
        \hline
        \hline

        \textbf{eq} & $0b00$ & $==$ \\
        \hline

        \textbf{ne} & $0b01$ & $!=$ \\
        \hline

        \textbf{gt} & $0b10$ & $>$ (označeno) \\
        \hline

        \textbf{al} & $0b11$ & true (bezuslovno) \\
        \hline
    \end{tabu}
    \caption{Pregled uslovnih sufiksa}
    \label{table:2}

\end{table}

\noindent Dodatne napomene:
\begin{itemize}
    \item Sve aritmetičke operacije se izvode tako da odgovaraju označenim
          celim brojevima.
    \item Operacije pomeranja su aritmetička pomeranja.
    \item Instrukcije \textbf{cmp} i \textbf{test} ne čuvaju direktni rezultat
          odgovarajuće opera\-cije, nego samo u skladu sa rezultatom postavljaju
          nove vrednosti flegova u PSW.
    \item Asembler poseduje i pseudoinstrukciju \textbf{ret} (povratak
          iz potprograma) koja se prevodi odgovarajućom \textbf{pop}
          instrukcijom.
    \item Asembler poseduje i pseudoinstrukciju \textbf{jmp} (bezuslovni
          skok), koja se prevodi instrukcijom \textbf{mov} ili \textbf{add},
          u zavisnosti od tipa adresiranja odredišta.
\end{itemize}

\subsubsection{Načini adresiranja}

U poljima od 5 bitova za \textit{dst} i \textit{src} prva (viša)
dva bita kodiraju tip adresiranja. U tipovima adresiranja u kojima je potreban
registar, preostala 3 bita kodira\-ju registar. U slučaju da je za adresiranje
potreban i podatak koji predstavlja adresu, pomeraj ili neposrednu vrednost,
zapisuje se u poslednja dva bajta instrukcije (tada je instrukcija dugačka
4 bajta). U jednoj instrukciji samo jedan operand može da zahteva dva
dodatna bajta, u suprotnom je u pitanju greška.

\begin{table}[h!]
\centering
\small

    \begin{tabu}{ | X[l] | X[l] | X[l] | }
        \hline
        \cellcolor{blue!25}\textbf{\textit{Adresiranje}} &
        \cellcolor{blue!25}\textbf{\textit{Kod}} &
        \cellcolor{blue!25}\textbf{\textit{Objašnjenje}} \\
        \hline
        \hline

        Neposredno ili PSW & $0b00$ &
        Ukoliko je u preostala 3 bita zapisano 0x7 koristi se PSW,
        u suprotnom je u dodatna 2 bajta zapisan neposredni podatak. \\
        \hline

        Registarsko direktno & $0b01$ &
        Preostala 3 bita kodiraju registar koji se koristi. \\
        \hline

        Memorijsko direktno & $0b10$ &
        Preostala 3 bita se ne koriste, u dodatna 2 bajta zapisana je
        adresa. \\
        \hline

        Registarsko indirektno sa pomerajem & $0b11$ &
        Preostala 3 bita sadrže registar koji se koristi pri adresiranju,
        a u 2 dodatna bajta zapisan je označeni pomeraj. \\
        \hline
    \end{tabu}
    \caption{Pregled načina adresiranja}
    \label{table:3}
\end{table}

\subsection{Napomene}
\begin{itemize}
    \item Ovaj odeljak se najviše sastoji od teksta projektnog zadatka
          postavlje\-nog na sajt predmeta Sistemski softver, koji važi počev
          od juna 2018. godine.
\end{itemize}



\newpage
\section{Opis rešenja}

\subsection{Asembler}

\subsubsection{Lekser i parser}
Tokom procesa leksičke analize (tokenizacije) lekser klasifikuje tekst izvornog
koda u sledeće kategorije tokena:
\begin{flushleft}
\texttt{EOF, NEWLINE, COMMENT, DEC\_LITERAL, SYMBOL,\\
INSTRUCTION, DIRECTIVE, REGISTER, COLON, COMMA, PLUS, MINUS,\\
AMPERSAND, ASTERISK, LEFT\_BRACKET, RIGHT\_BRACKET, DOLLAR}
\end{flushleft}
Ukoliko postoje leksičke greške koje onemogućavaju ispravnu tokenizaciju koda,
biće prijavljene od strane leksera, nakon čega se obustavlja dalje prevo\-đenje
asemblerskog koda. Lekser ume i da prepozna situaciju u kojoj se poslednja linija
ulaznog fajla se izvornim kodom ne završava znakom za novi red -- tada se automatski
dodaje još jedan \texttt{NEWLINE} token kako bi svi redovi bili ispravno terminisani.

Nakon tokenizacije, parser započinje proces sinta\-ksne analize tokom koje grupiše
tokene u kompleksnije strukture, definisane speci\-fikacijom sintakse asemblerskog
jezika. Zahvaljujući relativno jednosta\-vnoj sintaksi, parser je implementiran kao
\textit{recursive descent parser}, koji se sastoji od uzaja\-mno rekurzivnih procedura,
od kojih svaka parsira jednu sintaksnu konstru\-kciju. Zbog ovakvog načina parsiranja,
implementacija par\-sera dosta podseća na definiciju jezika u nekoj od sintaksnih
notacija. Parser će prijaviti prvu sinta\-ksnu grešku na koju naiđe, nakon čega se
obustavlja dalje prevođe\-nje.\\

\noindent
Definicija asemblerskog jezika u EBNF sintaksnoj notaciji:
\begin{flushleft}
\texttt{MODULE = \{LINE\} EOF . \\
LINE = \lbrack LABEL\rbrack\ \lbrack COMMAND\rbrack\ \lbrack COMMENT\rbrack\
    NEWLINE . \\
LABEL = SYMBOL "{}:"{} . \\
COMMAND = MNEMONIC \lbrack PARAM \{"{},"{} PARAM\}\rbrack\ . \\
MNEMONIC = (INSTRUCTION | DIRECTIVE) . \\
PARAM = (LITERAL | SYMVAL | SYMBOL | MEMLIT | REGADR | PCREL) . \\
LITERAL = \lbrack ("{}+"{}|"{}-"{})\rbrack DEC\_LITERAL . \\
SYMVAL = "{}\&"{} SYMBOL . \\
MEMLIT = "{}*"{} LITERAL . \\
REGADR = REGISTER \lbrack "{}\lbrack"{}(LITERAL | SYMBOL)"{}\rbrack"{}\rbrack . \\
PCREL = "{}\$"{} SYMBOL . \\
}
\end{flushleft}
\newpage % necessary evil

\subsubsection{Struktura relokativnog objektnog fajla}
Asembler kao rezultat prevođenja generiše binarni relokativni objektni fajl, u dobro
definisanom formatu. Takođe, postoji opcija i generisanja tekstualnog relokativnog
objektnog fajla koji može na čitljiv način da prikaže sadržaj fajla. Format binarnog
relokativnog objektnog fajla je sledeći:
\begin{enumerate}
    \item Tabela simbola zapisana u sledećem formatu:
    \begin{enumerate}
        \item 32-bitni neoznačen ceo broj $k$ koji predstavlja broj ulaza u tabeli simbola
        \item $k$ zapisa (ulaza u tabeli simbola) odgovarajuće veličine koji formi\-raju tabelu simbola
    \end{enumerate}
    \item 32-bitni neoznačen ceo broj $r$ koji predstavlja broj tabela sa relokati\-vnim zapisima
    \item $r$ tabela sa relokativnim zapisima zapisanih u sledećem formatu:
    \begin{enumerate}
        \item 32-bitni neoznačen ceo broj $m$ koji predstavlja broj relokacionih zapisa u tabeli
        \item 16-bitni neoznačen ceo broj $s$ koji predstavlja broj ulaza u tabeli simbola
              sekcije na koju se odnosi relokacioni zapis
        \item $m$ relokacionih zapisa odgovarajuće veličine koji formiraju tabelu relokacionih zapisa
    \end{enumerate}
    \item Tabela sa zaglavljima sekcija odgovarajuće veličine koja sadrži informa\-cije o broju sekcija
          $n$ (maks. 4) i 4 zapisa koji daju informacije o broju ulaza sekcije u tabeli simbola,
          veličini sekcije i naznačenoj adresi za učitavanje sekcije u memoriju
   \item $n$ sekcija zapisanih u vidu niza bajtova koje formiraju relokativni obje\-ktni kod
\end{enumerate}
\newpage % necessary evil

\noindent
Na primeru potprograma napisanog u asemblerskom jeziku koji računa najve\-ći zajednički delilac
parametara prosleđenih kroz registre r1 i r2 prikazan je sadržaj binarnog relokativnog objektnog
fajla i izgled tekstualnog relokati\-vnog objektnog fajla.
\begin{lstlisting}
;gcd.s - Greatest Common Divisor (GCD)
.text
.global gcd
gcd:
            cmp r2, r1
            jmpeq stop
            jmpgt less
            sub r1, r2
            jmp gcd
less:       sub r2, r1
            jmp gcd
stop:       mov r0, r1
            ret
.end
\end{lstlisting}
Sadržaj binarnog relokativnog fajla objektnog, prikazan korišćenjem hex. editora \texttt{xxd}:
\begin{lstlisting}
00000000: 0500 0000 0000 0000 0000 0000 0000  ..............
0000000e: 0000 0000 0000 0000 0000 0000 0000  ..............
0000001c: 0000 0000 0000 0000 0000 0000 0000  ..............
0000002a: 0000 0000 0100 0100 0100 0000 0000  ..............
00000038: 2e54 4558 5400 0000 0000 0000 0000  .TEXT.........
00000046: 0000 0000 0000 0000 0000 0000 0000  ..............
00000054: 0000 0000 0200 0200 0100 0000 0100  ..............
00000062: 6763 6400 0000 0000 0000 0000 0000  gcd...........
00000070: 0000 0000 0000 0000 0000 0000 0000  ..............
0000007e: 0000 0000 0300 0200 0100 1000 0000  ..............
0000008c: 6c65 7373 0000 0000 0000 0000 0000  less..........
0000009a: 0000 0000 0000 0000 0000 0000 0000  ..............
000000a8: 0000 0000 0400 0200 0100 1600 0000  ..............
000000b6: 7374 6f70 0000 0000 0000 0000 0000  stop..........
000000c4: 0000 0000 0000 0000 0000 0000 0000  ..............
000000d2: 0000 0000 0100 0000 0400 0000 0100  ..............
000000e0: 0000 0000 0400 0100 0100 0000 0800  ..............
000000ee: 0100 0200 0000 0e00 0200 0300 0000  ..............
000000fc: 1400 0200 0100 0000 0100 c3bf c3bf  ..............
0000010a: 1a00 0000 0000 0000 0000 0000 0000  ..............
00000118: 0000 0000 0000 0000 0000 0000 0000  ..............
00000126: c391 4935 c3b0 1600 c2b5 c3b0 1000  ..I5..........
00000134: c385 2ac3 b5c3 b000 00c3 8549 c3b5  ..*........I..
00000142: c3b0 0000 c3b5 09c3 a9c3 a00a       ............
\end{lstlisting}
\newpage % necessary evil

\noindent
Izgled tekstualnog relokativnog objektnog fajla:
\begin{lstlisting}
### SYMBOL TABLE ###
+----------------------------------------------------+
| INDEX|   NAME|     TYPE|  SECTION|   VALUE|    BIND|
+----------------------------------------------------+
|     0|       |    UNDEF|        0|       0|   LOCAL|
|   0x1|  .TEXT|  SECTION|      0x1|       0|   LOCAL|
|   0x2|    gcd|   SYMBOL|      0x1|       0|  GLOBAL|
|   0x3|   less|   SYMBOL|      0x1|    0x10|   LOCAL|
|   0x4|   stop|   SYMBOL|      0x1|    0x16|   LOCAL|
+----------------------------------------------------+
### .TEXT RELOCATION TABLE ###
+----------------------------------------------------+
| INDEX| RELOCATION_TYPE|    OFFSET|           SYMBOL|
+----------------------------------------------------+
|     0|  REL_TYPE_ABS16|       0x4|              0x1|
|   0x1|  REL_TYPE_ABS16|       0x8|              0x1|
|   0x2|  REL_TYPE_ABS16|       0xe|              0x2|
|   0x3|  REL_TYPE_ABS16|      0x14|              0x2|
+----------------------------------------------------+
### SECTION HEADER TABLE ###
+----------------------------------------------------+
|          SECTION_INDEX|      SIZE|     LOAD_ADDRESS|
+----------------------------------------------------+
|                    0x1|      0x1a|           0xffff|
+----------------------------------------------------+
### .TEXT SECTION ###
d1 49 35 f0 16 00 b5 f0 10 00 c5 2a f5 f0 00 00 c5 49
f5 f0 00 00 f5 09 e9 e0
\end{lstlisting}
Tekstualni izlazni fajlovi su samo reprezentativni,
sa svrhom da na čitljiv način prikažu strukturu i sadržaj
relokativnih objektnih fajlova. Oni neće biti kasnije
korišćeni od strane linkera ili emulatora.

\subsubsection{Formiranje tabele simbola - prvi prolaz}
Prilikom prvog prolaza asemblera rade se provere na semantičke greške
(neo\-dgovarajući broj operanada instrukcija, nepravilni tipovi adresiranja...)
i formira se tabela simbola. Ukoliko asembler pronađe semantičku grešku,
prijaviće je i obustaviti proces prevođenja.

Direktiva \texttt{.global}
obrađuje se tokom prvog prolaza, sa ciljem da se obezbedi podrška za deklarisanje
simbola globalnim bilo gde u fajlu, ne obavezno pre prve upotrebe,
tako da je na početku drugog prolaza tabela simbola u potpunosti izgrađena.
Asembler ovde otkriva i greške višestrukog definisanja simbola, što uključuje
i višestru\-ko definisanje sekcija, jer nije dozvoljeno da se ista sekcija
pojavi više od jednom unutar jednog fajla.

\subsubsection{Generisanje koda - drugi prolaz}
Prilikom drugog prolaza asemblera formiraju se tabela sa zaglavljima sekcija,
relokacione tabele za svaku sekciju na osnovu korišćenih simbola i generiše
se relokativni objektni kod na osnovu sadržaja sekcija.

Tabela sa zaglavljima sekcija
sadrži informaciju o broju sekcija i 4 zagla\-vlja (za svaku potencijalnju
sekciju po jedno) koja sadrže informacije o ulazu koji sekcija zauzima u
tabeli simbola, veličini sekcije i adresi na koju sekcija treba da se učita.
Rezervisana vrednost \texttt{0xffff} označava da za sekciju nije specificirano od koje
adrese se učitava, tako da linker kasnije treba da odredi tu adresu.

Za svaku definisanu sekciju generiše se tabela sa relokacionim zapisima.
Za svaki simbol korišćen unutar sekcije za koji asembler ne zna vrednost
generisaće se relokacioni zapis u relokacionoj tabeli te sekcije.
Svaka reloka\-ciona tabela odnosi se na neku sekciju, a svaki relokacioni zapis
(ulaz u tabeli) sadrži informacije o tome na kom pomeraju u odnosu na početak
sekcije treba izvršiti relokaciju, tipu relokacije i ulazu u tabeli simbola
za simbol na koji se odnosi relokacija. Na pomenuti pomeraj u odnosu na početak
sekcije upisuje se označena celobrojna vrednost $p$ umesto vrednosti simbola.
Kada linker izgradi sopstvenu
tabelu simbola, vrednost $p$ će sabrati sa vrednošću koju izračuna na osnovu
tipa relokacije i upisati novodobijenu vrednost na pomeraj zadat u relokacionom zapisu.

Nakon što se generišu tabela sa zaglavljima sekcija, relokacione tabele i relokativni
objektni kod, u izlazni fajl se ispisuju svi podaci koji su definisani formatom
relokativnog objektnog fajla. Dobijeni izlazni fajl je nakon toga spreman za
povezivanje od strane linkera.

\subsubsection{Dodatne napomene}
\begin{itemize}
    \item Mnemonici koji predstavljaju instrukcije, direktive i mnemonike su
          \textit{case-insensitive}. Imena simbola (labela) su \textit{case-sensitive}.
    \item PC-relativno adresiranje dozvoljeno je koristiti samo u instrukcijama skoka
          i poziva potprograma.
    \item Registarsko indirektno adresiranje nije dozvoljeno koristiti u instrukci\-jama
          skoka i poziva potprograma (ovo je implicitno moguće jedino korišćenjem
          PC-relativnog adresiranja).
    \item Omogućeno je koristiti mnemonike instrukcija bez sufiksa za uslovno izvršavanje,
          što je ekvivalentno korišćenju mnemonika sa sufiksom za bezuslovno izvršavanje.
    \item Dodata je pseudoinstrukcija HALT, koja obustavlja izvršavanje pro\-grama.
    \item Korišćenje registarskog direktnog adresiranja u instrukcijama skoka i poziva
          potprograma je dozvoljeno, pri čemu će u PC biti upisana vrednost sadržana
          unutar naznačenog registra.
\end{itemize}

\subsection{Linker}

\subsubsection{Struktura izvršnog fajla}
Linker kao rezultat povezivanja generiše binarni izvršni fajl, u dobro defini\-sanom
formatu. Takođe, postoji opcija i generisanja tekstualnog izvršnog fajla, koji može
na čitljiv način da prikaže sadržaj fajla. Format binarnog izvršnog fajla je sledeći.
\begin{enumerate}
    \item Tabela simbola zapisana u sledećem formatu:
    \begin{enumerate}
        \item 32-bitni neoznačen ceo broj $k$ koji predstavlja broj ulaza u tabeli simbola
        \item $k$ zapisa (ulaza u tabeli simbola) odgovarajuće veličine koji formi\-raju tabelu simbola
    \end{enumerate}
    \item Tabela zaglavlja programa zapisana u sledećem formatu:
    \begin{enumerate}
        \item 32-bitni neoznačen ceo broj $s$ koji predstavlja broj ulaza u tabeli
              zaglavlja programa
        \item $s$ zapisa o segmentima koda odgovarajuće veličine koji formiraju tabelu zaglavlja programa.
              Svaki zapis nosi informacije o tome koji ulaz u tabeli simbola ogovara simbolu sekcije (segmenta),
              adresi učitavanja segmenta i veličini segmenta
    \end{enumerate}
    \item $s$ segmenata zapisanih u vidu niza bajtova koji formiraju objektni (mašinski) kod
\end{enumerate}

\noindent
Na primeru potprograma napisanog u asemblerskom jeziku koji računa najve\-ći zajednički delilac
parametara prosleđenih kroz registre r1 i r2 i glavnog programa, napisanog u odvojenom fajlu, koji poziva ovaj
potprogram, prika\-zan je sadržaj tekstualnog izvršnog fajla.
(izlaza linkera).
\newpage % necessary evil

\begin{lstlisting}
;main.s - GCD test program
.data
M:              .word 36
N:              .word 27
GCD:            .word
.text
.global gcd
.global START
START:
                mov r1, M
                mov r2, N
                call gcd
                mov GCD, r0
                halt
.end
\end{lstlisting}
Izgled tekstualnog relokativnog objektnog fajla \texttt{main.s}, gde je prilikom
ase\-mbliranja eksplicitno zadato da sekcije treba učitavati počev od adrese \texttt{0x200}:
\begin{lstlisting}
### SYMBOL TABLE ###
+----------------------------------------------------+
|  INDEX|  NAME|     TYPE|  SECTION|   VALUE|    BIND|
+----------------------------------------------------+
|      0|      |    UNDEF|        0|       0|   LOCAL|
|    0x1| .DATA|  SECTION|      0x1|       0|   LOCAL|
|    0x2|     M|   SYMBOL|      0x1|       0|   LOCAL|
|    0x3|     N|   SYMBOL|      0x1|     0x2|   LOCAL|
|    0x4|   GCD|   SYMBOL|      0x1|     0x4|   LOCAL|
|    0x5| .TEXT|  SECTION|      0x5|       0|   LOCAL|
|    0x6|   gcd|   SYMBOL|        0|       0|  GLOBAL|
|    0x7| START|   SYMBOL|      0x5|       0|  GLOBAL|
+----------------------------------------------------+
### .DATA RELOCATION TABLE ###
+----------------------------------------------------+
|  INDEX| RELOCATION_TYPE|   OFFSET|           SYMBOL|
+----------------------------------------------------+
+----------------------------------------------------+
### .TEXT RELOCATION TABLE ###
+----------------------------------------------------+
|  INDEX| RELOCATION_TYPE|   OFFSET|           SYMBOL|
+----------------------------------------------------+
|      0|  REL_TYPE_ABS16|      0x2|              0x1|
|    0x1|  REL_TYPE_ABS16|      0x6|              0x1|
|    0x2|  REL_TYPE_ABS16|      0xa|              0x6|
|    0x3|  REL_TYPE_ABS16|      0xe|              0x1|
+----------------------------------------------------+



### SECTION HEADER TABLE ###
+----------------------------------------------------+
|          SECTION_INDEX|      SIZE|     LOAD_ADDRESS|
+----------------------------------------------------+
|                    0x1|       0x6|            0x200|
|                    0x5|      0x14|            0x206|
+----------------------------------------------------+
### .DATA SECTION ###
24 00 1b 00 00 00 
### .TEXT SECTION ###
f5 30 00 00 f5 50 02 00 ee 08 00 00 f6 08 04 00 d8 e0 10 00 
\end{lstlisting}
Izgled tekstualnog izvršnog fajla dobijenog povezivanjem binarnih reloka\-tivnih
objektnih fajlova \texttt{gcd.o} i \texttt{main.o}:
\begin{lstlisting}
### SYMBOL TABLE ###
+---------------------------------------------------------+
|  INDEX|       NAME|     TYPE|  SECTION|   VALUE|    BIND|
+---------------------------------------------------------+
|      0|           |    UNDEF|        0|       0|   LOCAL|
|    0x1| .SEGMENT.1|  SECTION|      0x1|   0x100|   LOCAL|
|    0x2|        gcd|   SYMBOL|      0x1|   0x100|  GLOBAL|
|    0x3| .SEGMENT.2|  SECTION|      0x3|   0x200|   LOCAL|
|    0x4| .SEGMENT.3|  SECTION|      0x4|   0x206|   LOCAL|
|    0x5|      START|   SYMBOL|      0x4|   0x206|  GLOBAL|
+---------------------------------------------------------+
### PROGRAM HEADER TABLE ###
+---------------------------------------------------------+
|      SEGMENT_INDEX|               SIZE|     LOAD_ADDRESS|
+---------------------------------------------------------+
|                0x1|               0x1a|            0x100|
|                0x3|                0x6|            0x200|
|                0x4|               0x14|            0x206|
+---------------------------------------------------------+
### .SEGMENT.1 SECTION ###
d1 49 35 e0 16 01 b5 e0 10 01 c5 2a f5 e0 00 01 c5 49 f5 e0
00 01 f5 09 e9 e8 
### .SEGMENT.2 SECTION ###
24 00 1b 00 00 00 
### .SEGMENT.3 SECTION ###
f5 30 00 02 f5 50 02 02 ee 08 00 01 f6 08 04 02 d8 e0 10 00 
\end{lstlisting}

Više primera programa napisanih na asemblerskom jeziku može se naći
unutar foldera \texttt{examples/} koji se nalazi u korenskom direktorijumu
projekta.
\newpage % necessary evil

\subsubsection{Prvi prolaz - spajanje objektnih fajlova}
Linker u prvom prolazu sastavlja sopstvenu tabelu simbola, koja se sastoji od
simbola za sekcije (segmente) i globalnih eksportovanih (izvezenih) simbola svih fajlova
koje je dobio kao ulaz, spaja sadržaje sekcija svih fajlova u jedan radni fajl
i formira listu relokacionih zahteva sastavljenu čitajući relokacione
zapise iz relokacionih tabela pronađenih u ulaznim fajlovima.

Ukoliko je u relokativnom objektnom fajlu za sekciju zapisana adresa od koje je
treba učitati, linker će početak sekcije povezati sa tom adresom. U suprotnom,
linker učitava sekcije počev od unapred određene adrese, pritom svaku sekciju
za koju u relokativnom objektnom fajlu nije navedeno od koje adrese treba da
se učita će nadovezati na prethodnu takvu sekciju. Linker na ovaj način vezuje
početak svake sekcije za neku adresu, odnosno vrednost simbola koji predstavlja
sekciju (vrednost sekcije) postavlja na tu adresu.

Istovremeno, linker ažurira i vrednosti svih
globalnih eksportovanih si\-mbola u odnosu na novu vrednost sekcije kojoj simbol
pripada. U ovom momentu, ukoliko se dogodi takva situacija za neki simbol,
biće prijavljena greška višestrukog definisanja simbola.

Sekcije iz ulaznih fajlova prepisuju se u radni fajl u vidu \textbf{segmenata} sa dodeljenom
početnom adresom i veličinom. Termin \textit{sekcija} i \textit{segment} se ovde koriste
ekvivalentno, s tim što se sekcija odnosi na objektni kod u relokativnom objektnom fajlu
a segment na objektni kod u izvršnom fajlu (mašinski kod).

Lista relokacionih zahteva se sastavlja tako što se relokacioni zapisi pre\-pravljaju
koristeći ulaze u tabeli simbola linkera, umesto ulaza iz lokalnih tabela simbola
za ulazne fajlove.

\subsubsection{Drugi prolaz - prepravljanje objektnog koda}
U drugom prolazu, linker za svaki relokacioni zahtev pokuša da izvrši prepra\-vku
objektnog koda. Ukoliko nije definisan simbol na koji se odnosi relokacije, biće
prijavljena greška.

Nakon prepravke objektnog koda, u tabeli simbola proverava se da li je definisan
simbol START koji predstavlja adresu prve instrukcije programa.
Ukoliko nije, linker prijavljuje grešku.

Nakon toga, za sve segmente koda definisane sekcijama iz ulaznih fajlova proverava
se da li postoji preklapanje bilo koja dva segmenta. Ukoliko postoji, linker
prijavljuje grešku.

Radni fajl se, na kraju, ispisuje u izlazni fajl, koji je nakon toga spreman da bude
učitan od strane emulatora.

\subsubsection{Dodatne napomene}
\begin{itemize}
    \item Omogućena je opcija koja linkeru nalaže da ignoriše adrese za učitava\-nje
          sekcija zapisane u relokativnim objektnim fajlovima. U ovom slu\-čaju, linker
          samostalno raspoređuje sekcije po memoriji počevši od unapred određene
          adrese.
    \item Linker je u stanju da prepozna situaciju u kojoj je segmet previše veliki
          da bi se uspešno učitao od zadate početne adrese (nema dovoljno memorije
          do kraja adresnog prostora). U ovom slučaju biće prijavljena greška i
          prekinuće se proces povezivanja.
    \item Implementacija linkera očekuje na ulazu ispravne, neoštećene binarne
          relokativne objektne fajlove. Provere validnosti sadržaja ulaznih fajlova
          nisu realizovane.
\end{itemize}



\subsection{Emulator}

\subsubsection{Rad emulatora}
Emulator čita binarne izvršne fajlove generisane od strane linkera i svaku instrukciju
sprovodi kroz četiri faze: \textit{fetch}, \textit{decode}, \textit{execute} i
\textit{interrupt}. Pored emuliranja rada procesora, emulira se i memorija, rad ulaznog
i izlaznog uređaja, i hardverskog tajmera.

Emulator iz ulaznog fajla najpre pročita tabelu simbola i tabelu zaglavlja programa.
Nakon toga, na osnovu zapisa iz tabele zaglavlja simbola, čita segmente koda iz
ulaznog fajla i učitava ih u emuliranu memoriju počev od zadate adrese.

Adresa prve instrukcije koja se upisuje u PC je vrednost simbola START koju emulator
traži u tabeli simbola. Emulator zatim ciklično izvršava sledeći niz akcija:
\begin{enumerate}
    \item Dohvatanje instrukcije (\textit{fetch}): Sa adrese upisane u registar PC
          čita se instrukcija odgovarajuće veličine i smešta se u jedan ili dva
          instrukcijska registra veličine 2 bajta.
    \item Dekoduju se operandi instrukcije (\textit{decode}), i po potrebi se adresa
          ope\-randa u emuliranoj
          memoriji zapisuje u registar MAR (\textit{memory address register}). Ukoliko
          je detektovana nevalidna instrukcija, biće generisan prekid.
    \item Na osnovu instrukcijskog koda (nad operandima) se izvršava instrukcija
          \textit{execute}, i po potrebi upisuje rezultat.
    \item Ispituje se da li je došlo do prekida (\textit{interrupt}). Ukoliko jeste,
          poziva se odgovarajuća prekidna rutina, pod uslovom da prekidi nisu maskirani,
          i da je pokazivač na prekidnu rutinu validan (nije \textit{nullpointer}).
    \item Ukoliko je rezultat upisan u memoriju na adresu registra izlaznog ure\-đaja,
          šalje se signal izlaznom uređaju da je novi podatak za ispis spreman.
    \item Ispituje se ulazni uređaj radi provere da li je spreman novi ulazni podatak.
          Ukoliko jeste, biće generisan prekid.
    \item Ispituje se tajmer radi provere da li je došlo do periodičnog tajmerskog prekida.
\end{enumerate}
Ukoliko je pročitana instrukcija koja u programskoj statusnoj reči postavlja indikator
završetka programa (pseudoinstrukcija HALT u asemblerskom je\-ziku),
obustavlja se emuliranje programa.

\subsubsection{Dodatne napomene}
\begin{itemize}
    \item Implementacija emulatora očekuje na ulazu ispravan, neoštećen binarni izvršni
          fajl. Provera validnosti sadržaja ulaznog fajla nije realizovana.
    \item Podsistem za zaštitu memorije ne postoji. Program ima punu slobodu da u bilo
          koju lokaciju memorije upiše bilo koju vrednost.
    \item Ne postoji podsistem za pamćenje prekida. Uvek će biti obrađen samo poslednji
          generisan prekid u slučaju da se generiše više prekida u kra\-tkom vremenskom
          intervalu.
\end{itemize}



\newpage
\section{Uputstvo za prevođenje izvornog koda}
Kako bi se projekat uspešno preveo, potrebno je najpre preuzeti izvorni kod sa GitHub
\href{https://github.com/cicovic-andrija/ETF-System-Software}{repozitorijuma} kom se
može pristupiti preko sledećeg URL-a:\\
\url{https://github.com/cicovic-andrija/ETF-System-Software}
Jedan od načina na koji se može preuzeti izvorni kod je korišćenjem alata \texttt{git},
unutar željenog direktorijuma:
\begin{lstlisting}[frame=noline, backgroundcolor=\color{cyan}]
$ git clone https://github.com/cicovic-andrija/ETF-System-Software.git
\end{lstlisting}
Ukoliko ne postoji, treba napraviti direktorijum \texttt{bin/} unutar
\texttt{home/} direktori\-juma tekućeg korisnika:
\begin{lstlisting}[frame=noline, backgroundcolor=\color{cyan}]
$ mkdir -p ~/bin
\end{lstlisting}
Program se prevodi upotrebom alata \texttt{make}. Izvršavanjem komande:
\begin{lstlisting}[frame=noline, backgroundcolor=\color{cyan}]
$ make
\end{lstlisting}
prevode se asembler, linker, emulator od strane alata \texttt{gcc}, i generiše se
dokumentacija od strane alata \texttt{pdflatex}. Izvršni fajlovi alata smeštaju se
u direktorijum \texttt{./bin/} i kopiraju u \texttt{home/bin/} direktorijum tekućeg korisnika
(pod pretpostavkom da postoji).
Dokumentacija
u PDF formatu smešta se u direktorijum \texttt{./doc/pdf/}.

Moguće je i svaku komponentu
prevesti pojedinačno upotrebom jedne od narednih komandi:
\begin{lstlisting}[frame=noline, backgroundcolor=\color{cyan}]
$ make assembler
$ make linker
$ make emulator
$ make doc
\end{lstlisting}

\texttt{Makefile} unutar korenskog direktorijuma projekta je \textit{top-level} makefile,
koju upravlja makefile-ovima unutar direktorijuma \texttt{assembler/}, \texttt{linker/},
\texttt{emulator/} i \texttt{doc/}. Izvršavanjem komande:
\begin{lstlisting}[frame=noline, backgroundcolor=\color{cyan}]
$ make
\end{lstlisting}
unutar jednog od ovih direktorijuma prevodi se odgovarajuća komponenta proje\-kta.

Pravilo \texttt{clean} definisano je unutar svakog makefile-a. Upotrebom ovog pravila
brišu se neželjeni fajlovi unutar stabla direktorijuma projekta.

Nakon prevođenja, alati su spremni za upotrebu od strane programera. Ukoliko se asembler
ili linker pozovu uz korišćenje opcije \texttt{-h} biće prikazano kratko korisničko
uputstvo. Emulator se poziva sa jednim ulaznim fajlom; ukoliko se pozove bez ulaznih
fajlova, ispisaće kratko korisničko uputstvo.

